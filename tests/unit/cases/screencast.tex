\screencast[X5ArSbUea_o]{html-class-id-attributes}{Inspecting the ID and Class attributes}%
{ch_arch/mov/html-class-id-attributes.mp4}{%
CSS uses \w[CSS selector]{selector notations} such as \T{div\#}\emph{name} to
indicate a \T{div} element whose \T{id} is \emph{name} and
\T{div.}\emph{name} to indicate a \T{div} element with class \emph{name}.
Only one element in an HTML document can have a given \T{id}, whereas many
elements (even of different tag types) can share the same \T{class}.
All three aspects of an element---its tag type, its \T{id} (if it has
one), and its \T{class} attributes (if it has any)---can be used to
identify an element as a candidate for visual formatting.
}

\screencast[yX1tMdBuG3g]{haml}{Interpolation into views using Haml}{ch_arch/mov/haml.mp4}{%
\index{Interpolation, template views|textit}%
\index{HTML Abstraction Markup Language (Haml)!template views|textit}%
In a Haml template, lines
beginning with \T{\%} expand into the corresponding HTML opening tag,
with no closing tag needed since Haml uses indentation to determine
structure.  Ruby-like hashes following a tag become HTML attributes.
Lines \mbox{\C{--beginning with a dash}} are executed as Ruby code with
the result discarded, and lines \mbox{\C{=beginning with an equals
sign}} are executed as Ruby code with the result interpolated
into the HTML output.
}

\screencast[X5ArSbUea_o]{html-class-id-attributes}{Inspecting the ID and Class attributes}%
{ch_arch/mov/html-class-id-attributes.mp4}{%
CSS uses \w[CSS selector]{selector notations} such as \T{div\#}\emph{name} to
indicate a \T{div} element whose \T{id} is \emph{name} and
\T{div.}\emph{name} to indicate a \T{div} element with class \emph{name}.
Only one element in an HTML document can have a given \T{id}, whereas many
elements (even of different tag types) can share the same \T{class}.
All three aspects of an element---its tag type, its \T{id} (if it has
one), and its \T{class} attributes (if it has any)---can be used to
identify an element as a candidate for visual formatting.
}\index{Selector notations, CSS|textit}
