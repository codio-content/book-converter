The goal of this exercise is to translate a recursive definition into a Java method.
The Ackermann function is defined for non-negative integers as follows:
\begin{eqnarray*}
A(m, n) = \begin{cases}
              n+1 & \mbox{if } m = 0 \\
              A(m-1, 1) & \mbox{if } m > 0 \mbox{ and } n = 0 \\
              A(m-1, A(m, n-1)) & \mbox{if } m > 0 \mbox{ and } n > 0
\end{cases}
\end{eqnarray*}

Write a recursive method called \java{ack} that takes two \java{int}s as parameters and that computes and returns the value of the Ackermann function.

For example, the {\bf factorial} of an integer $n$, which is written $n!$, is defined like this:
%
\begin{eqnarray*}
&&  0! = 1 \\
&&  n! = n \cdot(n-1)!
\end{eqnarray*}
%
Don't confuse the mathematical symbol $!$, which means {\em factorial}, with the Java operator \java{!}, which means {\em not}.

\begin{eqnarray*}
& \sin \frac{\pi}{4} + \frac{\cos \frac{\pi}{4}}{2} & \\
& \log 10 + \log 20 &
\end{eqnarray*}
