\begin{concepts}

\w{JavaScript} is a dynamic, interpreted scripting language built into
modern browsers.
This chapter describes its main features, including some that we
recommend avoiding
because they represent questionable design choices, and how it extends
the types of content and applications that can be delivered as SaaS.

\begin{itemize}

\item A browser represents a web
page as a data structure called the \w{Document Object Model} (DOM). \js{}
code running in the browser can inspect and modify this data
structure, causing the browser to redraw the modified page elements.

\item When a user interacts with the browser (for example, by
typing, clicking, or moving the mouse) or the browser makes progress
in an interaction with a server, the browser generates an
\w[Event (computing)]{event} indicating what happened.  Your
\js{} code can take app-specific actions to modify the DOM when such
events occur.

\item Using \w[Ajax (programming)]{AJAX}, or Asynchronous \js{} And XML,
\js{} code can make
HTTP requests to a Web server \emph{without} triggering a page
reload.  The information in the response can then be used to modify page
elements in place, giving a richer and often more responsive
user experience than traditional
Web pages.  Rails partials and controller actions
can be readily used to handle AJAX interactions.

\item Just as we use the highly-productive Rails framework
(Chapter~\ref{chap:rails_intro}) and RSpec
TDD tool (Chapter~\ref{chap:tdd}) for server-side SaaS code, here we use
the highly-productive \w{jQuery} framework and
\weblink{http://pivotal.github.com/jasmine}{Jasmine}
TDD tool to develop
client-side code.

\item We follow the best practice of ``graceful degradation,'' also
referred to as ``progressive enhancement'':
legacy browsers lacking \js{} support will still provide a good
user experience, while \js-enabled browsers will provide an even
better experience.

\end{itemize}

\end{concepts}
