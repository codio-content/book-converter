\begin{enumerate}
\item Type in the hello world program, then compile and run it.

\item Add a print statement that displays a second message after the ``Hello, World!''.
Say something witty like, ``How are you?''
Compile and run the program again.

\item Add a comment to the program (anywhere), recompile, and run it again.
The new comment should not affect the result.
\end{enumerate}

Recall that the hope for plan-and-document methods is to make software engineering as predictable in budget and schedule as civil engineering. Remarkably, user stories, points, and velocity correspond to \emph{seven} major tasks of the plan-and-document methodologies. They include:
\index{Plan-and-Document!major tasks}%
\begin{enumerate}
    \item Requirements Elicitation
    \item Requirements Documentation
    \item Cost Estimation
    \item Scheduling and Monitoring Progress
\end{enumerate}
\index{Waterfall lifecycle!tasks}\index{Spiral lifecycle!tasks}%
\index{Rational Unified Process (RUP)!tasks}%
These are done up front for the Waterfall model and at the beginning of each major iteration for the Spiral and RUP models. As requirements change over time,  these items above imply other tasks:
\begin{enumerate} \addtocounter{enumi}{4}
\item Change Management for Requirements, Cost, and Schedule
\item Ensuring Implementation Matches Requirement Features
\end{enumerate}
Finally, since accuracy of the budget estimate and the schedule is vital to the success of the plan-and-document process, there is another task not found in BDD:
\begin{enumerate} \addtocounter{enumi}{6}
\item Risk Analysis and Management
\end{enumerate}
The hope is that by imagining all the risks to the budget and schedule in advance, the project can make plans to avoid or overcome them.
