\begin{elaboration}{Versions of Agile}
There is not just a single Agile lifecycle. We are following \w{Extreme Programming} (XP), which includes one- to two-week iterations, behavior driven design (see Chapter~\ref{chap:bdd}), test-driven development (see Chapter~\ref{chap:tdd}), and pair programming (see Section \ref{sec:Pair}). Another popular version is \w[Scrum (software development)]{Scrum} (see Section \ref{sec:Scrum}), where self-organizing teams use two- to four-week iterations called \x{sprints}, and then regroup to plan the next sprint.
  \index{Scrum}%
  \index{Sprint, Scrum}%
A key feature is daily standup meetings to identify and overcome obstacles. While there are multiple roles in the scrum team, the norm is to rotate the roles over time. The \w[Kanban (development)]{Kanban} approach is derived from Toyota's just-in-time manufacturing process, which in this case treats software development as a pipeline.
  \index{Kanban}%
Here the team members have fixed roles, and the goal is to balance the number of team members so that there are no bottlenecks with tasks stacking up waiting for processing. One common feature is a wall of cards that to illustrate the state of all tasks in the pipeline. There are also hybrid lifecycles that try to combine the best of two worlds. For example, \x{ScrumBan} uses the daily meetings and sprints of Scrum but replaces the planning phase with the more dynamic pipeline control of the wall of cards from Kanban.
  \index{ScrumBan}%
\end{elaboration}
  \index{Agile lifecycle!variants}%
  \index{Extreme Programming (XP)!Agile variants}%

  \index{Software acquisition, DOD}%
  \index{Department of Defense (DOD), software acquisition}%
  \index{DOD|see {Department of Defense (DOD)}}%