We can therefore say that the expression
\C{3+2} results in calling \C{Fixnum\#+} on the receiver \C{3}.

\tablefigure{ch_ruby_intro/tables/syntactic_sugar_table.tex}{fig:sugar}{%
  The first column is Ruby's syntactic sugar for common operations, the
  second column shows the explicit method call, and the third column
  shows how to perform the same method call using Ruby's \C{send}, which
  accepts either a string or (more idiomatically) a symbol for the method name.
}

\tablefigure{ch_intro/tables/ACA}{fig:ACA}{%
Comparing Amazon.com and Healthcare.gov during its first three
months. (\cite{Thorp13}) After its stumbling start, the deadline was
extended from December 15, 2013 to March 31, 2014, which explains the
lower goal in customers per day in December. Note that availability for
ACA does \emph{not} include time for ``scheduled maintenance,'' which
Amazon does include (\cite{Zients13}). The error rate was for significant errors on the forms sent to insurance companies (\cite{Horsley13}). The site was widely labeled by security experts as insecure, as the developers were under tremendous pressure to get proper functionality, and little attention was paid to security (\cite{Harrington13}).}

\tablefigure{ch_ruby_rails/tables/ruby_collection_methods}{fig:functional} {%
   Some common Ruby methods on collections.  For those that expect a block,
   the ``Block'' column shows the number of arguments expected by the block; if
   blank, the method doesn't expect a block.  For
   example, a call to \C{sort}, whose block expects 2 arguments, might look like: \C{c.sort~\{~\textbar{}a,b\textbar{}~a~$<=>$~b~\}}.
   These methods all return a new object rather than modifying the
   receiver, but some methods also have a \emph{destructive} variant ending in
   \C{!}, for example \C{sort!}, that modify their argument in place  (and also return the new value).
   Use destructive methods with extreme care, if at all.
  }
