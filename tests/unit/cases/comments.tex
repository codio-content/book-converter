Many people (and textbooks) incorrectly refer to \java{\%} as the ``modulus operator''.
In mathematics, however, {\bf modulus} is the number you're dividing by.
In the previous example, the modulus is 12.
The Java language specification refers to  \java{\%} as the ``remainder operator''.

The remainder operator looks like a percent sign, but you might find it helpful to think of it as a division sign ($\div$) rotated to the left.

%Note that both \java{/} and \java{\%} perform {\em integer division}, so the result always rounds down.
%The reason why integer division ``rounds down'' is that the hardware computes the quotient and remainder separately.

\index{divisible}
\index{extract digits}

Modular arithmetic turns out to be surprisingly useful.
For example, you can check whether one number is divisible by another: if \java{x \% y} is zero, then \java{x} is divisible by \java{y}.
You can use remainder to ``extract'' digits from a number: \java{x \% 10} yields the rightmost digit of \java{x}, and \java{x \% 100} yields the last two digits.
And many encryption algorithms use the remainder operator extensively.


\section{Putting it all together}

At this point, you have seen enough Java to write useful programs that solve everyday problems.
You can (1) import Java library classes, (2) create a \java{Scanner}, (3) get input from the keyboard, (4) format output with \java{printf}, and (5) divide and mod integers.
Now we will put everything together in a complete program:

%Since we've looked at each of these topics in isolation, it's important to see how they fit together in a complete program.
%If you've been working through the examples on your computer as you've been reading (like we recommended in Section~\ref{sec:examples}), then good job!


Using division and modulo, we can convert to feet and inches like this:

\begin{code}
    feet = 76 / 12;    // quotient
    inches = 76 % 12;  // remainder
\end{code}

The first line yields 6.
The second line, which is pronounced ``76 mod 12'', yields 4.
So 76 inches is 6 feet, 4 inches.

Addition, subtraction, and multiplication all do what you expect, but you might be surprised by division.
For example, the following fragment tries to compute the fraction of an hour that has elapsed:%, but it has a logic error:

At this point, you have seen enough Java to write useful programs that solve everyday problems.
You can (1) import Java library classes, (2) create a \java{Scanner}, (3) get input from the keyboard, (4) format output with \java{printf}, and (5) divide and mod integers.
Now we will put everything together in a complete program:

%Since we've looked at each of these topics in isolation, it's important to see how they fit together in a complete program.
%If you've been working through the examples on your computer as you've been reading (like we recommended in Section~\ref{sec:examples}), then good job!

\index{Convert.java}

\begin{trinket}{Convert.java}
import java.util.Scanner;

/**
* Converts centimeters to feet and inches.
*/
public class Convert {

    public static void main(String[] args) {
        double cm;
        int feet, inches, remainder;
        final double CM_PER_INCH = 2.54;
        final int IN_PER_FOOT = 12;
        Scanner in = new Scanner(System.in);

        // prompt the user and get the value
        System.out.print("Exactly how many cm? ");
        cm = in.nextDouble();

        // convert and output the result
        inches = (int) (cm / CM_PER_INCH);
        feet = inches / IN_PER_FOOT;
        remainder = inches % IN_PER_FOOT;
        System.out.printf("%.2f cm = %d ft, %d in\n",
        cm, feet, remainder);
    }
}
\end{trinket}

Although not required, all variables and constants are declared at the top of \java{main}.
This practice makes it easier to find their types later on, and it helps the reader know what data is involved in the algorithm.
