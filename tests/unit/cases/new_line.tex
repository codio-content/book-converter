\begin{quote}
No bottles of beer on the wall,\\
no bottles of beer,\\
ya' can't take one down, ya' can't pass it around,\\
'cause there are no more bottles of beer on the wall!
\end{quote}

\begin{itemize}
\item On Linux, it's likely under: \verb"/usr/lib/jvm/openjdk-.../"
\\ If not, then install the {\tt openjdk-...-source} package.
\item On MacOS, it's likely under: \\ \verb"/Library/Java/JavaVirtualMachines/jdk.../Contents/Home/"
\item On Windows, it's likely under: \verb"C:\Program Files\Java\jdk...\"
\end{itemize}

\begin{itemize}

\item Ellen Hildreth used this book to teach Data Structures at Wellesley College and submitted a whole stack of corrections, along with some great suggestions.

\item Tania Passfield pointed out that some glossaries had leftover terms that no longer appeared in the text.

\item Elizabeth Wiethoff noticed that the series expansion of $\exp(-x^2)$ was wrong.
She has also worked on a Ruby version of the book.

\item Matt Crawford sent in a whole patch file full of corrections.

\item Chi-Yu Li pointed out a typo and an error in one of the code examples.

\item Doan Thanh Nam corrected an example.

\item Muhammad Saied translated the book into Arabic, and found several errors in the process.

\item Marius Margowski found an inconsistency in a code example.

\item Leslie Klein discovered another error in the series expansion of $\exp(-x^2)$, identified typos in the card array figures, and gave helpful suggestions to clarify several exercises.

\item Micah Lindstrom reported half a dozen typos and sent corrections.

\item James Riely ported the textbook source from LaTeX to Sphinx.
\\ \url{http://fpl.cs.depaul.edu/jriely/thinkapjava/}

\item Peter Knaggs ported the book to C\#.
\\ \url{http://www.rigwit.co.uk/think/sharp/}

\item Heidi Gentry-Kolen recorded several video lectures that follow the book.
\\ \url{https://www.youtube.com/user/digipipeline}

\item Waldo Ribeiro submitted a pull request that corrected a dozen typos.
\end{itemize}

Unlike string literals, which appear in double quotes, character literals can only contain a single character.
Escape sequences, like \java{'\\t'}, are legal because they represent a single character.

The increment and decrement operators also work with characters.
