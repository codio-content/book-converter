\codefilefigure[3ed03cadbcfc3982105c]{ch_javascript/code/json_example.js}{fig:json_example}{%
  \js{} notation for object literals, that is, objects you specify by
  enumerating their properties and values explicitly.
  If the property name is a legal \js{} variable name, quotes can be
  omitted or the idiomatic dot-notation shortcut (lines 13--14) can be
  used, although quotes are always required around all strings when an
  object is expressed in JSON format.
  Since objects can contain other
  objects, hierarchical data structures can be built (line 5) and
  traversed (lines 13--15).
}

\codefilefigure{ch_ruby_rails/code/class_example.rb}{fig:class_example}{%
  A simple class definition in Ruby showing that explicit getter and
  setter methods are the only way to access instance variables from
  outside a class, and that Ruby provides shortcuts (lines
  19--20) that avoid having to define every accessor method
  explicitly.  Rather than distinguish ``private'' vs.\ ``public'' instance and class
variables, one simply provides public accessor methods
(read-only, write-only, or read/write) for state that should be
publicly visible.
}

\codefilefigure{ch_ruby_rails/code/class_example.rb}{fig:class_example}  {%
A simple class definition in Ruby showing that explicit getter and
setter methods are the only way to access instance variables from
outside a class, and that Ruby provides shortcuts (lines
19--20) that avoid having to define every accessor method
explicitly.  Rather than distinguish ``private'' vs.\ ``public'' instance and class
variables, one simply provides public accessor methods
(read-only, write-only, or read/write) for state that should be
publicly visible.
}
