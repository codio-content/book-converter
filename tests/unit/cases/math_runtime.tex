\[ \runtime = a + b n \]

where $a$ is the intercept of the line and $b$ is the
slope.

On the other hand, if \java{add} is linear, the total time for
$n$ adds would be quadratic. If we plot runtime versus problem
size, we expect to see a parabola. Or mathematically, something like:

\[ \runtime = a + b n + c n^2 \]

With perfect data, we might be able to tell the difference between a
straight line and a parabola, but if the measurements are noisy, it can
be hard to tell. A better way to interpret noisy measurements is to plot
runtime and problem size on a \textbf{log-log} scale.
