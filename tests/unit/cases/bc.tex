\B{Summary of legacy code exploration}

\C{Voucher}

jQuery defines a global function \C{jQuery()} (aliased as \C{\$()})
that, when passed a CSS selector (examples of which we saw in
Figure~\ref{fig:css_cheat}), returns all of the current page's DOM
elements matching that selector.  For example,
\C{jQuery('\#movies')}
or \C{\$('\#movies')} would return the single element whose ID is
\T{movies}, if one exists on the page;
\C{\$('h1.title')} would return all the \T{h1} elements whose CSS class
is \T{title}.
A more general version of this functionality is
\C{.find(}\emph{selector}\C{)},
which only searches the DOM subtree rooted at the target.  To illustrate
the distinction, \C{\$('p~span')} finds \emph{any} \T{span} element that
is contained inside a \T{p} element, whereas if \C{elt} already refers
to a \emph{particular} \T{p} element, then \C{elt.find('span')} only
finds \T{span} elements that are descendants of \C{elt}.
\begin{sidebar}[-0.5in]{}%
    The call \C{jQuery.noConflict()} ``undefines'' the \C{\$} alias,
    in case your app uses the browser's built-in \C{\$} (usually an alias
    for \C{document.\-getElementById}) or loads another \js{} library such as
    \weblink{http://prototypejs.org}{Prototype} that also tries to define
    \C{\$}.
\end{sidebar}
