\index{Cascading Style Sheets (CSS)!overview|textbf}%
\index{HyperText Markup Language (HTML)!overview|textbf}%
\index{Application architecture!HTML and CSS|textbf}%
\label{sec:10k}
\label{sec:html}
\label{sec:css}


%% xhtml vs. html5:
%%

In contrast to a native app, which is designed to render a particular user
interface associated with only one SaaS service,
we can think of a desktop or mobile browser as a \emph{universal
client}, because any site the browser visits can deliver all the
information necessary to render that site's user interface.
Both browsers and native apps are used by millions of people, so we
call them \emph{production clients}\index{Production clients}.


Indeed, modern practice suggests that even when creating a user-facing SaaS app designed to be used
via a browser, we should design the app as a collection of resources accessible via RESTful APIs,
but then provide a Web browser-based user interface ``on top of'' those API calls.


If the Web browser is the universal client, \w{HTML}, the
HyperText Markup Language, is the
universal language.
A \w{markup language}\index{Markup language}
combines text with markup (annotations about the text) in a way
that makes it easy to
syntactically distinguish the two.

HTML consists of a hierarchy of nested elements,
each of which consists of an opening tag such as \T{\tl{}p\tg},
a content part (in some cases), and a closing tag such as \T{\tl/p\tg}.
Most opening tags can also have attributes, as in
\T{$<$a~href="http://\ldots"$>$}.
Some tags that don't have a content part are self-closing, such as
\T{\tl{}br clear="both"/\tg} for a line break that clears both left and
right margins.
\index{HyperText Markup Language (HTML)!introduction|textit}
% start RP home: ``we're going to look under hood''
% ensure Web Dev Toolbar enabled in FFox (first time mention)
% View Source; this is what comes back from server
% Markup language: easy to distinguish text from annotations
% Tags, eg <title>; must have open+close, or else self-closing, eg <link>
% everything inside the tags is the Element.
% eg Title elements' content is ``RP''
% Scroll down to <table> - it's a big element with nested thead/tbody,
% in turn with <tr> and  <th>/<td>
% ATTRIBUTES; table id=movies.  other attributes are allowed.
%  eg, in 'a' element, 'href' attribute is link target, but element
%  content is link text, as in ``More about Inception''

  \begin{sidebar}{}%
  The use of angle brackets for tags
  comes from \w{SGML} (Standard Generalized Markup Language),
  a codified standardization of IBM's Generalized Markup
  Language, developed  in the 1960s for encoding
  computer-readable project documents.
  \end{sidebar}
  \index{General Markup Language|textit}%
  \index{Standard Generalized Markup Language (SGML)|textit}%

There is an unfortunate and confusing mess of terminology surrounding the
\weblink{http://www.w3.org/TR/html5/introduction.html\#history-1}{lineage of HTML}
\index{HyperText Markup Language (HTML)!HTML 5 features}
HTML 5 includes features of both its predecessors
(HTML versions 1 through 4) and XHTML\index{eXtended HyperText Markup Language (XHTML)!HTML 5}
(eXtended HyperText Markup Language)\index{eXtensible Markup Language (XML)!HTML 5},
which is a subset of \w{XML}, an eXtensible Markup Language
that can be used both to represent data and to describe other markup
languages.  Indeed, XML is a common data representation for exchanging
information \emph{between} two services in a Service-Oriented
Architecture,
as we'll see in
Chapter~\ref{chap:tdd} when we extend RottenPotatoes to retrieve
movie information from a separate movie database service.  The
differences among the variants of XHTML and HTML are difficult to keep
straight, and not all browsers support all versions.  Unless otherwise
noted, from now on when we say HTML we mean
HTML~5, and we will try to avoid using features that aren't widely
supported.

%% \codefilefigure[2b3e750c9389d9266634]{ch_arch/code/movies.xml}{fig:xml}%
%%  {SaaS applications
%%   commonly exchange information encoded in XML, such as this possible
%%   representation of a list of movies.}

Of particular interest are the HTML tag attributes \T{id} and \T{class},
because they figure heavily into connecting the HTML structure of a
page with its visual appearance.  The
following screencast illustrates the use of Firefox's Web Developer
toolbar to quickly identify the ID's and Classes of HTML elements on a
page.

\screencast[X5ArSbUea_o]{html-class-id-attributes}{Inspecting the ID and Class attributes}%
{ch_arch/mov/html-class-id-attributes.mp4}{%
CSS uses \w[CSS selector]{selector notations} such as \T{div\#}\emph{name} to
indicate a \T{div} element whose \T{id} is \emph{name} and
\T{div.}\emph{name} to indicate a \T{div} element with class \emph{name}.
Only one element in an HTML document can have a given \T{id}, whereas many
elements (even of different tag types) can share the same \T{class}.
All three aspects of an element---its tag type, its \T{id} (if it has
one), and its \T{class} attributes (if it has any)---can be used to
identify an element as a candidate for visual formatting.
}\index{Selector notations, CSS|textit}
% start: homepage + view source
% in last screencast, we noticed some tags have attributes, such as
% h1.title
%  Use FFox 'tools > display > element info'; move it out of way
%  when hover over element, it gets outlined in red (including any
%  nested elt), and shows nesting in toolbar.
% two attribute are especially significant: id and class.
%  ex: 'table#main'.  '#' means 'ID attribute'; here it is in HTML.
%  ex: 'h1.title'   '.' means 'class attribute'.
% next screencast, we'll see how class and ID attrs can be used to tell
% browser how to associate visual style attributes with different elements.

\begin{sidebar}[0.7in]{}%
For an extreme example of how much can be done with CSS, visit the
  \weblink{http://csszengarden.com}{CSS Zen Garden}.
\end{sidebar}
\index{CSS Zen Garden|textit}%
As the next
screencast shows, the \w{CSS} (\w{Cascading Style Sheets}) standard
allows us to associate visual ``styling'' instructions with HTML
elements by using the elements' classes and IDs.
The screencast covers
only a few basic CSS constructs, which are summarized in
Figure~\ref{fig:css_cheat}.  The Resources section at the end of the
chapter lists sites and books
that describe CSS in  great detail, including how to use CSS for aligning
content on a page, something designers used to do manually with HTML tables.



\screencast[E5ZVorHn_fs]{css-intro}{Introduction to CSS}{ch_arch/mov/css-intro.mp4}{%
  There are four basic mechanisms by which a selector in a CSS file can
  match an HTML element: by tag name, by class, by ID, and by hierarchy.
  If multiple selectors match a given element, the rules for which
  properties to apply are complex, so most designers try to avoid such
  ambiguities by keeping their CSS simple.  A useful way to see the
  ``bones'' of a page is to select \Sf{CSS\tg\-Disable Styles\tg\-All
    Styles} from the Firefox Web Developer toolbar; most
  developer-friendly browsers offer a ``developer mode'' featuring similar behaviors.
  Disabling styles will display
  the page with all CSS formatting turned off, showing the extent to
  which CSS can be used to separate visual appearance from logical
  structure.
}
\index{Cascading Style Sheets (CSS)!introduction|textit}%
% open page in FFox, open Element Inspector and move it out of the way
% show link tag in head element associates css stylefile with this page
%  (each page must reference the style file(s) it wants)
%  open by clicking on link
%  in Rails, lives in app/assets/stylesheets/ subdirectory
% 4 ways to match elements to formatting, or 'selectors'
% tag example: use 'Display element info' in FFox to show h2
%  1) style based on 'tag' - All Movies is an h1.
%  2) ID: div#main - must be unqiue - has extra padding
%  3) class: h1.title - overrides regular h1 because more specific.
%  4) nesting: table#movies th  (ps, more specific than just 'th')
%   if no explicit match, inherit from enclosing tag
% full ref: tag backgrounds can be images, repeating patterns, etc.


\tablefigure{ch_arch/tables/css_cheat}{fig:css_cheat}{%
   A few CSS constructs, including those explained in
   Screencast~\ref{css-intro}.  The top table shows some CSS
   \emph{selectors}, which identify the elements to be styled; the
   bottom table shows a few of the many attributes, whose names are
   usually self-explanatory, and example values
   they can be assigned.  Not all attributes are valid on all elements.
}
\index{Selector notations, CSS|textit}%
\index{Cascading Style Sheets (CSS)!constructs|textit}%

%% \begin{elaboration}{Tables != Formatting}
%%   Before HTML+CSS, the only way to align chunks of content
%%   was to use HTML tables.  This practice is now deprecated because
%%   mixing content and visual information clutters the page (especially
%%   for accessibility software used by persons with disabilities), because
%%   it complicates changing the page's visual design, and because the same effects can
%%   now be achieved
%%   using CSS properties such as \T{float}.
%% \end{elaboration}

Using this new information,
Figure~\ref{fig:10k}  expands steps 2 and 3 from the previous section's
summary of how SaaS works.

\picfigure{ch_arch/figs/saas10k.pdf}{fig:10k}{%
  SaaS from 10,000 feet.  Compared to Figure~\ref{fig:50k}, step 2 has
  been expanded to describe the content returned by the Web server, and
  step 3 has been expanded to describe the role of CSS in how the Web
  browser renders the content.
}
\index{Web servers!content rendering|textit}%
\index{Cascading Style Sheets (CSS)!content rendering|textit}%
\index{Web browsers!content rendering|textit}%
\index{Software as a Service (SaaS)!content rendering|textit}%

\begin{NEW}

  CSS provides for sophisticated layout behaviors, but can be tricky
  to use in 2 regards. First, some background in layout and graphic
  design is helpful in deciding how to style site elements - spacing,
  typography, color palette. Second, even if you know what you want
  the site to look like, it can be tricky to write the necessary CSS
  to achieve complex layouts, in which elements may "float" to the far
  left or far right of the page while text flows around them, or
  rearrange themselves responsively when the screen geometry changes
  (browser window resized, phone rotates) to provide a defensible user
  experience on both desktop and mobile devices.

Front-end frameworks use two mechanisms to "package" both aesthetic
choices about visual styling and the coding required to achieve that
styling.

CSS classes - define complex behaviors, provide reasonable default
choices about typography and layout. By following certain rules about
page structure - for example, which elements should be nested inside
which others - and attaching the appropriate CSS classes to the
elements, a variety of common element layouts can be achieved.

JavaScript - more sophisticated behaviors such as animations,
collapsing menus, and drag-and-drop also require some JavaScript
code. In many cases, you need never directly call this code: instead,
you annotate page elements with yet other CSS classes, and the
framework arranges to trigger the correct JavaScript code when the
user interacts with elements having those classes. (Section [binding
  JavaScript to the DOM] explains the mechanics of how this is done.)

Rather than delving deeply into the aesthetics of graphic design, our
goal for you as a well-rounded full-stack developer is that you should
know how to provide well-structured pages with proper element nesting
and clean CSS class tags, for two reasons. First, with a good
front-end framework, just doing this will be enough to provide an AURA
site (aesthetic, usable, responsive, accessible). Second, a clean site
layout and classing allows designers to refine, customize, or work
separately on the site's visual appearance.

There are 2 principles to using front-end frameworks successfully in
this way: semantic styling and grid layout. Semantic styling means
First, think of your page's visual elements not in terms of their
visual characteristics ("This message should appear in a red box with
bold text") but in terms of their function ("This message signals an
error", "This text is the page title", "This list of items is a menu
of navigation options"). A good front-end framework names its styles
according to the function they enable an element to fulfill, rather
than according to their visual appearance.


\end{NEW}

\begin{elaboration}{SPA or MPA?}
SPA vs MPA: Are you building something that's more like a website
(transactional? lots of different possible screens? user navigates
large amounts of data in a typical session? multi-screen workflows?)
or more like a desktop app (few screens? continuous interaction vs
transactional, eg something like Pivotal Tracker? user navigates
modest amounts of data in a typical session? short workflows or
interactions typically limited to 1 screen?)  If it's primarily a
website, use HTML5 + jQuery where needed. If primarily an app, may be
better off using a framework. [Need a checklist like "when to use
  agile" for "when to build a SPA vs MPA"].  Examples of popular SPAs:
Gmail, Google Docs, Pivotal Tracker.  Popular MPAs: IMDb, Amazon.com,
Google Search.  *MPA vs SPA is primarily a user experience question,
not a technical one!*

\end{elaboration}


\begin{summary}
\B{Summary}
  \begin{itemize}
  \item An \w{HTML} (HyperText Markup Language) document consists
    of a hierarchically nested collection
    of elements. Each
    element begins with a  \w[Html tag]{tag} in \tl{}angle
    brackets\tg{} that may have optional
    \w[HTML\#Attributes]{attributes}.  Some elements enclose content.
  \item A \w[CSS selector]{selector} is an expression that identifies
    one or more HTML elements in a document by using a combination of
    the element name (such as \T{body}), element \T{id}
        (an element attribute that must be unique on a page), and
    element \T{class} (an attribute that need not be unique on a page).
  \item \w{Cascading Style Sheets} (CSS) is a stylesheet language describing
    visual attributes of elements on a Web page.  A stylesheet associates
    sets of visual properties with  selectors.  A
    special \T{link} element inside the \T{head} element of an HTML
    document associates a stylesheet with that document.
  \item The ``developer tools'' in each browser, such as the Firefox Web
    Developer toolbar, are invaluable in peeking under
    the hood to examine both the structure of a page and its stylesheets.
  \end{itemize}
\end{summary}

\begin{checkyourself}
  True or false: every HTML element must have an ID.

  \begin{answer}
  False---the ID is optional, though must be unique if provided.
  \end{answer}
\end{checkyourself}

\begin{sidebar}[-0.5in]{GitHub Gists}
make it easy to copy-and-paste the code.
\ifhtmloutput
  The link accompanying each code example will take you to that code
  example on GitHub.
\else
  You need to type in the URI if you're reading the print book.
\fi
\end{sidebar}
\index{GitHub Gists|textit}%

\begin{checkyourself}
  \label{ex:css1}
  Given the following HTML markup:

  \codefile[f1336efcfafef83c8c51]{ch_arch/code/htmlexercise.html}
  Write down a CSS selector that will select \emph{only} the word
  \emph{Mondays} for styling.
  \begin{answer}
  Three possibilities, from most specific to least specific, are:
  \C{\#i~span}, \C{p.x~span}, and \C{.x~span}.
  Other selectors are possible but redundant or over-constrained; for
  example,
  \C{p\#i~span} and \C{p\#i.x~span}
  are redundant with respect to this HTML snippet since at
  most one element can have the ID \C{i}.
  \end{answer}
\end{checkyourself}

\begin{checkyourself}
  In Self-Check \ref{ex:css1}, why are \C{span}
  and  \C{p~span} \emph{not} valid answers?
  \begin{answer}
  Both of those selector also match \emph{Tuesdays}, which is a
  \C{span} inside a \C{p}.
  \end{answer}
\end{checkyourself}

\begin{checkyourself}
  What is the most common way to associate a CSS stylesheet with an HTML
  or HTML document? (HINT: refer to the earlier screencast example.)

  \begin{answer}
    Within the \T{HEAD} element of the HTML or HTML document, include a
    \T{LINK} element with at least the following three attributes:
    \T{REL="STYLESHEET"}, \T{TYPE="text/css"},
    and \T{HREF="\emph{uri}"}, where \T{\emph{uri}}
    is the full or partial URI of  the stylesheet.  That is, the
    stylesheet must be accessible as a resource named by a URI.
  \end{answer}
\end{checkyourself}
